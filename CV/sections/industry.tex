% \textbf{Nanjing Huawei}.
% After graduating from Nanjing University in 2017, I joined Nanjing Huawei and
% worked for a year in the department of Data Center Switches. I was responsible for
% developing and maintaining the third-layer functionality and VXLAN feature of
% the switch. During this period, I gained a lot of experience in the challenges
% of software development. I then decide to pursue a higher degree and joined
% the Ph.D. program at HKUST for software engineering and security.

\smallskip
\textbf{Tencent Keen Lab}.
During my Ph.D. study, I had a half-year internship in Keen Lab and did my
research on binary similarity analysis. The work \texttt{sem2vec}, published
on TOSEM 2023, is a result of this internship.
% \texttt{sem2vec}
% extracts semantic information from binary code to help developers understand
% the functionality of malicious addons.
I was awarded the \textbf{CCF-Tencent Rhino-Bird Elite Award} for this work.

% \smallskip
% \textbf{OpenArkCompiler}.
% I leaded our team to realize the address sanitizer (ASAN) for Huawei's
% OpenArkCompiler. In this project, besides the fundamental instrumentation
% for ASAN, we also implemented the runtime library, a subsitution of LLVM's ASAN
% runtime library. Moreover, we also developed several improvements to
% accelerate the execution performance of ASAN-protected software.
% The repository is available at
% \href{https://gitee.com/openarkcompiler/OpenArkCompiler/tree/master/src/mapleall/maple\_san}{here}.
